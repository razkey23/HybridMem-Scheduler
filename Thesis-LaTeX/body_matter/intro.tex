\chapter{Εισαγωγή} 
\InitialCharacter{Τ}α τελευταία χρόνια παρατηρείται εκτεταμένη διείσδυση της Μηχανικής Μάθησης σε κάθε κλάδο. Η Μηχανική μάθηση αποτελεί παράγοντα καινοτομίας σε ένα μεγάλο φάσμα εφαρμογών που περιλαμβάνει απο εμπορικά προιόντα έως και ιατρικές εφαρμογές. Η ευρύτητα αυτού του φάσματος ,σε συνδυασμό με το γεγονός ότι η απόδοση των μοντέλων που αναπτύσσονται με τεχνικές μηχανικής μάθησης είναι αρκετά καλή μοντελοποιώντας εφαρμογές που παραδοσιακά θα ήταν αρκετά περίπλοκο να μοντελοποιηθούν, ωθούν στην ραγδαία ανάπτυξη της. Ταυτόχρονα, στον κλάδο της αρχιτεκτονικής υπολογιστών η προόδος που προβλέπεται απο το νόμο του Moore φαίνεται ότι σταδιακά παύει να ακολουθεί το εκθετικό μοτίβο αύξησης που ακολουθούσε έως τώρα , ενώ το χάσμα επιδόσεων μεταξύ μνήμης και επεξεργαστή δεν εχει γεφυρωθεί ακόμα. Οι δύο αυτές τάσεις ,δηλαδή η εξέλιξη της Μηχανικής Μάθησης και τα ίδια τα προβλήματα που υπάρχουν στην Αρχιτεκτονικη Υπολογιστων, ωθούν προς μια συνδυαστική αξιοποιήση μεθόδων και τεχνικών , έτσι ώστε η μηχανική μάθηση να υποστηρίζεται αλλα κυρίως να υποστηρίζει την αρχιτεκτονική.


\begin{comment}
\InitialCharacter{Ο} Παγκόσμιος Ιστός test αποτελεί χώρο διακίνησης τεράστιου όγκου
πληροφοριών. Ωστόσο, η συντριπτική πλειοψηφία των πληροφοριών του
Ιστού, είναι προσανατολισμένη προς τον άνθρωπο-χρήστη και δεν
είναι κατανοητή από τις εφαρμογές. Για να αξιοποιηθεί λοιπόν η
διαθέσιμη πληροφορία και να γίνει πιο εύκολη η ανταλλαγή και η
επεξεργασία της, ο Παγκόσμιος Ιστός εξελίσσεται στο Σημασιολογικό
Ιστό.

Ο Σημασιολογικός Ιστός, είναι μια εξέλιξη του σημερινού Ιστού,
μέσα στον οποίο δίνεται καλά ορισμένο νόημα στην πληροφορία που
διακινείται, διευκολύνοντας τη συνεργασία μεταξύ υπολογιστή και
ανθρώπου \cite{LiArTs13}. Πιο συγκεκριμένα δίνει τη
δυνατότητα καλύτερης πρόσβασης σε μεγάλο όγκο πηγών πληροφορίας,
καθώς και πιο αποτελεσματικής διακίνησης των πληροφοριών,
χρησιμοποιώντας δεδομένα που τις περιγράφουν και ονομάζονται
$``$μεταδεδομένα$"$. Η καλύτερη γνώση της σημασίας, της χρήσης και
της ποιότητας των πηγών διευκολύνει σημαντικά τη δυνατότητα
πρόσβασης σε πηγές του Ιστού και την αυτόματη επεξεργασία του
περιεχομένου που υπάρχει διαθέσιμο στο Διαδίκτυο βάσει του
νοήματος και όχι μόνο της μορφής της πληροφορίας.

Ένα από τα πιο βασικά θέματα για την ανάπτυξη του Σημασιολογικού
Ιστού είναι το να μπορούν οι υπολογιστές να ανταλλάσσουν δεδομένα
 μεταξύ εφαρμογών. Σε ένα ανοιχτό περιβάλλον όπως
είναι ο Σημασιολογικός Ιστός χρειάζεται ένα ευέλικτο και δυναμικό
μοντέλο ανταλλαγής δεδομένων όπως είναι τα συστήματα ομότιμων
κόμβων \en{(Peer-to-Peer systems)}.

Ένα σύστημα ομότιμων κόμβων αποτελείται από ένα σύνολο αυτόνομων
υπολογιστικών κόμβων, οι οποίοι συνεργάζονται με σκοπό την
ανταλλαγή δεδομένων. Τα συστήματα ομότιμων κόμβων που
χρησιμοποιούνται ευρέως σήμερα κυρίως για την ανταλλαγή αρχείων
μουσικής, έχουν πολύ μικρές δυνατότητες διαχείρισης δεδομένων. Η
αναζήτηση πληροφορίας στα περισσότερα από αυτά γίνεται με χρήση
λέξεων κλειδιών \en{(keyword-based search)}.

Η ανάγκη για πιο εκφραστικές λειτουργίες, σε συνδυασμό με την
ανάπτυξη του Σημασιολογικού Ιστού, οδήγησε στα συστήματα ομότιμων
κόμβων που είναι βασισμένα σε σχήματα \en{(schema-based
peer-to-peer systems)}. Στα συστήματα αυτά κάθε κόμβος
χρησιμοποιεί ένα σχήμα με βάση το οποίο οργανώνει τα τοπικά
διαθέσιμα δεδομένα. Οι τεχνολογίες του Σημασιολογικού Ιστού δίνουν
τη δυνατότητα οργάνωσης των δεδομένων μέσω σχημάτων που τα
περιγράφουν.

Το πλαίσιο \en{RDF} είναι ένα τέτοιο εργαλείο αναπαράστασης
μεταδεδομένων. Σε ένα \en{RDF} αρχείο ορίζονται δηλώσεις για
αντικείμενα του Ιστού όπως σελίδες, συγγραφείς, προγράμματα κ.τ.λ.
Μια επέκταση του πλαισίου \en{RDF} είναι το \en{RDF Schema} το
οποίο παρέχει μηχανισμούς περιγραφής σχετικών αντικειμένων του
Ιστού καθώς και των σχέσεων μεταξύ τους. Το \en{RDF Schema}
βασίζεται σε κλάσεις και ιδιότητες έννοιες γνωστές από το χώρο των
Αντικειμενοστρεφών συστημάτων. Η βασική διαφορά είναι ότι στο
πλαίσιο \en{RDF} οι ιδιότητες ορίζονται ανεξάρτητα από τις
κλάσεις.

Χρησιμοποιώντας λοιπόν τις τεχνολογίες του Σημασιολογικού Ιστού
μπορούμε να δημιουργήσουμε συστήματα ομότιμων κόμβων με αυξημένη
διαλειτουργικότητα τα οποία θα ανταλλάσσουν μεταξύ τους πληροφορία
με νόημα και θα έχουν τη δυνατότητα διατύπωσης ερωτήσεων πιο
εκφραστικών από αυτές που βασίζονται σε λέξεις κλειδιά.
\end{comment}

\section{Αντικείμενο της διπλωματικής}
Η διπλωματική εργασία αυτή ασχολείται με τη χρήση τεχνητών νευρωνικών δικτύων για την χρονοδρομολόγηση σελίδων μνήμης σε σύγχρονα συστήματα που διαθέτουν δύο είδη κύριας μνήμης , τόσο συμβατική \en{DRAM} όσο και ένα σχετικά νέο είδος μνήμης την \en{Persistent Memory}

Είναι γνωστό ότι τα σύγχρονα υπολογιστικά συστήματα σχεδιάζονται χρησιμοποιώντας ετερογενή συστατικά μνήμης. Αυτές οι μνήμες συχνά  είτε εξυπηρετούν στην αύξηση της χωρητικότητας της κύριας μνήμης, δηλαδή σαν επέκταση της \en{DRAM}, είτε αξιοποιούνται ως κρυφές-μνήμες(\en{caches}) της κύριας μνήμης. Αυτά τα υβριδικά συστήματα μνήμης απαιτούν εκ φύσεως κάποιους σχεδιαστικούς συμβιβασμούς. Συνήθως η μακρύτερη στην ιεραρχία μνήμη, δηλαδή μνήμη που βρίσκεται πιο μακρία απο την Επεξεργαστική Μονάδα, έχει μεγαλύτερη χωρητικότητα αποθήκευσης αλλά ταυτόχρονα έχει και μεγαλύτερη καθυστέρηση (\en{latency}) και μειωμένο εύρος ζώνης (\en{bandwidth}).

Στη συγκεκριμένη διπλωματική εργασία θα περιοριστούμε σε υπολογιστικά συστήματα τα οποία αξιοποιούν την \en{Persistent Memory} ως επέκταση της κύριας μνήμης. Για το σχεδιασμό αυτών των συστημάτων είναι ιδιαίτερα σημαντικό να ληφθούν υπόψιν το μεγαλύτερο \en{latency} και το μειωμένο \en{bandwidth} που παρουσίαζει αυτή η μνήμη σε σχέση με τη \en{DRAM}. Στη συγκεκριμένη περίπτωση αυτό που μας ενδιαφέρει η αποδοτική υλοποιήση ενός Χρονοδρομολογητή Σελίδων Μνήμης (\en{Page Scheduler}) , δηλαδή η αποδοτική υλοποίηση της μονάδας που αναλαμβάνει τη διαχείριση της μνήμης του λειτουργικού συστήματος αλλά και τον εν εκτελέση προγραμμάτων. Ο χρονοδρομολογητής σελίδων θα είναι υπεύθυνος για τη μεταφορά σελίδων μνήμης απο και προς τα διάφορα ετερογενή συστατικά μνήμης που απαρτίζουν το σύστημα μας . Κύριο σκοπό θα έχει οι σελίδες μνήμης που θεωρούνται \textit{\en{Hot}}, δηλαδή προσπελαύνονται συχνά, να βρίσκονται στα υψηλής απόδοσης στοιχεία μνήμης που διαθέτει το σύστημα μας, δηλαδή στη \en{DRAM}, ενώ λιγότερο σημαντικές σελίδες μνήμης \textit{\en{Cold}} να βρίσκονται στην \en{Persistent Memory}. 

\section{Κίνητρο και συναφείς προσεγγίσεις}
Πολλοί ερευνητές έως τώρα έχουν αποπειραθεί να δώσουν μια λύση στο πρόβλημα της κατηγοριόποισης των σελίδων μνήμης και στην κατάλληλη τοποθέτηση τους στα διάφορα ετερογενή στοιχεία μνήμης. Πρόκειται σίγουρα για μία δύσκολη διαδικασία καθώς πρέπει να ληφθούν υπόψιν και το μοτίβο προσπέλασης μνήμης που ακολουθεί μία εφαρμογή όσο και οι παράμετροι εκτέλεσης της εφαρμογής (μέγεθος του \en{input}, \en{strong/weak scaling etc.}). Οι περισσότεροι ερευνητές έχουν προτείνει λύσεις στο παραπάνω πρόβλημα που μπορούν να ενσωματωθούν στο επίπεδο του \en{hardware} , των \en{Compilers} ,του Λειτουργικού Συστήματος ,του περιβάλλοντος εκτέλεσης. \cite{Chou2017BATMANTF,10.1145/2901318.2901344,10.1145/2926697.2926702,10.1145/3132402.3132418,10.1145/3241624.2926702,10.5555/3291656.3291698,DBLP:journals/corr/WuHL17}. Συχνά οι ερευνητές σε αυτές τις προσεγγίσεις χρησιμοποιούν πληροφορίες που σχετίζονται μονάχα με το ιστορικό προσπέλασης των σελίδων μνήμης. Συγκεκριμένα οι σύγχρονες τεχνικές που χρησιμοποιούνται στη δυναμική διαχείριση σελίδων σε επίπεδο συστήματος χρησιμοποιούν την πρόσφατη παρατηρούμενη συμπεριφορά των σελίδων προκειμένου να παρθούν αποφάσεις την μελλοντική τοποθέτηση τους.

\section{Προσέγγιση και Συνεισφορές}
Σε αυτή τη διπλωματική εργασία θα ακολουθηθεί η συλλογιστική πορεία των άρθρων \emph{\en{Learning Memory Access Patterns}} \cite{hashemi2018learning} και \emph{\en{Kleio: A hybrid Memory Page Scheduler}} \cite{10.1145/3307681.3325398}
\\

Σκοπός μας είναι η μελέτη και κατασκευή ενός Χρονοδομολογητή Σελίδων Μνήμης (\en{Hybrid Memory Page Scheduler}) χρησιμοποιώντας Μηχανική Μάθηση, ο  οποίος θα επιτυγχάνει καλύτερη απόδοση απο τις σύγχρονες μεθόδους οι οποίες στηρίζονται αποκλειστικά στην ιστορική παρατήρηση προσβάσεων στη μνήμη των εφαρμογών. Θα προσπαθήσουμε να απαντήσουμε σε ερωτήσεις που αφορούν το πως θα πετύχουμε μια αποδοτική λύση , δηλαδή πως θα καταφέρουμε να τοποθετήσουμε όσο πιο ιδανικά γίνεται τα σωστά δεδομένα σε επίπεδο σελίδων (\emph{\textbf{4KB}}) στις σωστές θέσεις. Ενώ ταυτόχρονα αναζητούμε και μια εφικτή λύση, χρησιμοποιώντας περιορισμένους επεξεργαστικούς πόρους για την κατασκευή των μοντέλων μηχανικής μάθησης που θα χρειαστούμε. Ακουλουθεί η αναλυτική περιγραφή των ζητημάτων που θα μελετηθούν σε αυτήν διπλωματική
\\
\\
\begin{itemize}
    \item Το χάσμα απόδοσης που υπάρχει στις σύγχρονες προσπάθειες αντιμετώπισης του προβλήματος. Λόγω της πρόσφατης άφιξης των νέων αυτών μνημών , ο τρόπος που αντιμετωπίζεται το πρόβλημα είναι αρκετά απλοϊκός με αποτέλεσμα να μην είναι αρκετά αποδοτικός όταν οι εφαρμογές προς εκτέλεση παρουσιάζουν΄περίπλοκα μοτίβα προσβάσεων μνήμης.
    \item Η κατάλληλη επιλογή της συχνότητας του Χρονοδρομολογητη. Θα διαπιστώσουμε αναλύοντας τις προσβάσεις μνήμης διάφορων μετροπρογραμμάτων κατά πόσο επηρεάζει την απόδοση η συχνότητα με την οποία ο Χρονοδρομολογητης αποφασίζει να μεταφέρει σελίδες μεταξύ των διάφορων στοιχείων μνήμης του συστήματος
    \item Χρονοδρομολογηση με χρήση Μηχανικής Μάθησης. Θα διαπιστώσουμε ότι τα επαναληπτικά Νευρωνικά Δίκτυα (\en{RNN}) μπορούν να αξιοποιηθούν κατάλληλα για το πρόβλημα του \en{Scheduling} όπως άλλωστε παρατηρήθηκε και στα άρθρα  \cite{hashemi2018learning,10.1145/3307681.3325398}. Θα προσπαθήσουμε να προσαρμόσουμε το πρόβλημα όσο πιο κατάλληλα γίνεται ώστε να επιτύχουμε ακριβείς προβλέψεις των Νευρωνικών Δικτύων χωρίς όμως να αγνοούμε την υπολογιστική και χωρική πολυπλοκότητα της προσέγγισης μας.
    \item Κατασκευή \en{Page-Scheduler} . Θα προσπαθήσουμε αναλογιζόμενοι διάφορες κρίσιμες μετρικές να κατασκευάσουμε έναν \en{Page-Scheduler} ο οποίος θα χρησιμοποιεί τόσο τις σύγχρονες μεθόδους \en{History-Based} Χρονοδρομολόγησης,όσο και Τεχνητή Νοημόσυνη βασισμένη στα Επαναληπτικά Νευρωνικά Δίκτυα προκειμένου να επιτευχθεί μια αύξηση της απόδοσης.
\end{itemize}




\section{Οργάνωση του τόμου}

\begin{comment}
Η εργασία αυτή είναι οργανωμένη σε επτά κεφάλαια: Στο Κεφάλαιο 2
δίνεται το θεωρητικό υπόβαθρο των βασικών τεχνολογιών που
σχετίζονται με τη διπλωματική αυτή. Αρχικά περιγράφονται τα δίκτυα
ομότιμων κόμβων, στη συνέχεια το πλαίσιο \en{RDF} και τέλος
δίνεται μια μελέτη των γλωσσών ερωτήσεων για \en{RDF}. Στο
Κεφάλαιο 3 αρχικά περιγράφονται οι σχετικές με το θέμα εργασίες
και στη συνέχεια δίνεται ο στόχος της συγκεκριμένης εργασίας. Στο
Κεφάλαιο 4 παρουσιάζεται η ανάλυση και η σχεδίαση του συστήματος,
δηλαδή η περιγραφή των υποσυστημάτων και των εφαρμογών του. Η
περιγραφή της υλοποίησης του συστήματος, με ανάλυση των βασικών
αλγορίθμων καθώς και λεπτομέρειες σχετικά με τις πλατφόρμες και τα
προγραμματιστικά εργαλεία που χρησιμοποιήθηκαν δίνεται στο
Κεφάλαιο 5. Στο Κεφάλαιο 6 παρουσιάζεται ο έλεγχος καλής
λειτουργίας του συστήματος με βάση ένα συγκεκριμένο σενάριο
χρήσης. Τέλος στο Κεφάλαιο 7 δίνεται η συνεισφορά αυτής της
διπλωματικής εργασίας, καθώς και μελλοντικές επεκτάσεις.
\end{comment}