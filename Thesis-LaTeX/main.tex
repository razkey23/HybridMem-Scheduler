%%%%%%%%%%%%%%%%%%%%%%%%%%%%%%%%%%%%%%%%%%%%%%%%%%%%%
%% File: main.tex
%% Author: Evangelos Stamos (estamos@e-ce.uth.gr)
%% Last update: January, 2020
%% Description: Provides an example of a Diploma Thesis 
%% using the ntua-thesis pdfLaTeX class.
%%
%% Character encoding: UTF-8
%%%%%%%%%%%%%%%%%%%%%%%%%%%%%%%%%%%%%%%%%%%%%%%%%%%%%
%
%
%%%%%%
% 1. use the "modern" or "classic" option to switch between 
% a modern or classic font, respectively.
%
% 2. add/remove the "hyperref" option to enable/disable hyperlinks:
% (remember to remove auxiliary files after adding/removing 
% the "hyperref" option).
%
% 3. add/remove the "printer" option to typeset a printer-friendly 
% (grayscale)/color version of the thesis.
%
% 4. use the "watermark" option to indicate that this is not an actual
% thesis.
%
% 5. use the "histinit" option to enable "historiated initials".
% (If used, all chapter initials declared by the \InitialCharacter{}
% macro are enlarged. If omitted, arguments of \InitialCharacter{}
% are typeset as normal text.)
%
% 6. use the "plain" option to disable tikz graphics in title page
% and part/chapter headers (might help to avoid compilation timeouts).
% Note that "plain" disables CD label and CD cover creation.
%
% 7. use the "noindex" option to (hopefully) avoid compilation timeouts
% when compiling online (disables index generation - note that "\indexGR",
% "\indexEN" invocations need not be removed when toggling this option).
%
% 8. activate the "newlogo" option to use the new official Logo.
%
%%%%%%%%%%%%%%%%%%%%%%%%%%%%%%%%%%%%%%%%%%%%%%%%%%%%%%%%%%%%%%%%%%%%%%%%%%%%%%%
%
\documentclass[modern,hyperref,watermark,histinit,noindex,plain,newlogo]{ntua-thesis}
%
%%%%%%%%%%%%%%%%%%%%%%%%%%%%%%%%%%%%%%%%%%%%%%%%%%%%%%%%%%%%%%%%%%%%%%%%%%%%%%%
%
%
%%%%%%%%%%%%%%%%%%%%%%%%%%%%%%%%%%%%%%%%%%%%%%%%%%%%
%% THESIS INFO 
%%%%%%%%%%%%%%%%%%%%%%%%%%%%%%%%%%%%%%%%%%%%%%%%%%%%
%
% ΤΙΤΛΟΣ ΔΙΠΛΩΜΑΤΙΚΗΣ ΕΡΓΑΣΙΑΣ 
%
% Για εξαναγκασμένες αλλαγές γραμμής χρησιμοποιήστε "\\".
% Αν οι αλλαγές γραμμής πρέπει να είναι διαφορετικές στο εξώφυλλο σε σχέση 
% με το εσώφυλλο (σελ. 3), επαναλάβετε τον τίτλο του εξωφύλλου με τις 
% επιθυμητές αλλαγές γραμμής ως προαιρετικό όρισμα της εντολής \title.
%
% Παραδείγματα:
% 1. Όμοιος τίτλος σε εξώφυλλο και εσώφυλλο, με αυτόματες αλλαγές γραμμής:
%	    \title{Πρότυπο Σύστημα Ομότιμων Κόμβων Βασισμένο σε Σχήματα \en{RDF}}
% 2. Όμοιος τίτλος σε εξώφυλλο και εσώφυλλο, με αλλαγή γραμμής μετά τη λέξη
% "Σύστημα":
%	    \title{Πρότυπο Σύστημα \\ Ομότιμων Κόμβων Βασισμένο σε Σχήματα \en{RDF}}
% 3. Διαφορετικές αλλαγές γραμμής σε εξώφυλλο και εσώφυλλο. Στο εξώφυλλο 
% έχουμε αλλαγή γραμμής μετά τη λέξη "Σύστημα", ενώ στο εσώφυλλο η αλλαγή
% γραμμής ακολουθεί τη λέξη "Ομότιμων":
%	    \title[Πρότυπο Σύστημα \\ Ομότιμων Κόμβων Βασισμένο %
%           σε Σχήματα \en{RDF}]% (προαιρετικό όρισμα)
%           {Πρότυπο Σύστημα Ομότιμων \\ Κόμβων Βασισμένο σε %
%           Σχήματα \en{RDF}}% (υποχρεωτικό όρισμα)
%
    \title{\el{Χρονοδρομολόγηση Σελίδων σε Υβριδικά Συστήματα Μνήμης με Χρήση Μοντέλων Μηχανικής Μάθησης}}           
	%\title{Πρότυπο Σύστημα Ομότιμων Κόμβων \\Βασισμένο σε Σχήματα \en{RDF}}
%%
%
%% -------------------------------------------------------------------
%% ΥΠΟΤΙΤΛΟΣ ΔΙΠΛΩΜΑΤΙΚΗΣ ΕΡΓΑΣΙΑΣ (προαιρετικός)
%
% Αν δεν υπάρχει υπότιτλος, τοποθετήστε τον χαρακτήρα του σχολίου "%"
% πριν από την εντολή \subtitle, ή αφήστε κενό το όρισμα της εντολής.
%
% Παράδειγμα:
%%	\subtitle{Μελέτη και υλοποίηση}
	\subtitle{Μελέτη και υλοποίηση}
%
%% -------------------------------------------------------------------
%% ΤΟΥ/ΤΗΣ/ΤΩΝ
%
% "του" ή "της" ή "των", ανάλογα με το φύλο/αριθμό του σπουδαστή ή 
% των σπουδαστών
% Παράδειγμα:
%	\toutis{του}
	\toutis{του}
%
%% -------------------------------------------------------------------
%% ΟΝΟΜΑΤΕΠΩΝΥΜΟ ΣΠΟΥΔΑΣΤΗ ΣΤΑ ΕΛΛΗΝΙΚΑ (ΚΕΦΑΛΑΙΑ, ΓΕΝΙΚΗ ΠΤΩΣΗ)
%
% Για περισσότερους του ενός σπουδαστές, διαχωρίστε με ",".
% Παράδειγμα:
%	\authorNameCapitalGR{ΚΩΝΣΤΑΝΤΙΝΟΥ Δ. ΔΗΜΗΤΡΙΟΥ, ΓΕΩΡΓΙΟΥ Π. ΠΑΝΑΓΑΚΗ}
	\authorNameCapitalGR{ΣΤΑΥΡΑΚΑΚΗ ΚΩΝΣΤΑΝΤΙΝΟΥ}
%
%% -------------------------------------------------------------------
%% ΟΝΟΜΑΤΕΠΩΝΥΜΟ ΣΠΟΥΔΑΣΤΗ ΣΤΗ ΛΑΤΙΝΙΚΗ ΜΟΡΦΗ (ΠΕΖΑ)
%
% Δηλώστε εδώ τυχόν ονοματεπώνυμα στη λατινική μορφή, αλλιώς αφήστε
% κενό το όρισμα.
% Για περισσότερους του ενός σπουδαστές, διαχωρίστε με ",".
% Παράδειγμα:
%	\authorNameEN{Albert Einstein, George W. Bush} 
	%\authorNameEN{Albert Einstein} 
%
%% -------------------------------------------------------------------
%% ΟΝΟΜΑΤΕΠΩΝΥΜΟ ΣΠΟΥΔΑΣΤΗ ΣΤΑ ΕΛΛΗΝΙΚΑ (ΠΕΖΑ, ΟΝΟΜΑΣΤΙΚΗ ΠΤΩΣΗ)
%
% Για περισσότερους του ενός σπουδαστές, διαχωρίστε με ",".
% Αν τα ονοματεπώνυμα όλων των σπουδαστών είναι σε λατινική μορφή,
% αφήστε κενό το όρισμα.
% Παράδειγμα:
%	\authorNameGR{Κωνσταντίνος Δημητρίου, Γεώργιος Παναγάκης}
	\authorNameGR{Σταυρακάκης Κωνσταντίνος}
%
%% -------------------------------------------------------------------
%% ΟΝΟΜΑΤΕΠΩΝΥΜΟ ΕΠΙΒΛΕΠΟΝΤΑ ΚΑΘΗΓΗΤΗ
% 
	\supervisor{\en{Supervisor}}
%
%% -------------------------------------------------------------------
%% ΤΙΤΛΟΣ ΕΠΙΒΛΕΠΟΝΤΑ ΚΑΘΗΓΗΤΗ
%
	\supervisorTitle{\en{Supervisor's Title}}
%
%% -------------------------------------------------------------------
%% ΕΠΙΒΛΕΠΩΝ/ΕΠΙΒΛΕΠΟΥΣΑ
%
% "Επιβλέπων" ή "Επιβλέπουσα", ανάλογα με το φύλο του 
% Επιβλέποντα Καθηγητή
	\supervisorMaleFemale{Επιβλέπων}
%
%% -------------------------------------------------------------------
%% ΤΟΠΟΣ/ΜΗΝΑΣ/ΕΤΟΣ ΕΚΔΟΣΗΣ
%
	\thesisPlaceDate{Αθήνα, Φεβρουάριος 2021}
%
%% -------------------------------------------------------------------
%% ΤΟΠΟΣ/ΜΗΝΑΣ/ΕΤΟΣ ΣΥΓΓΡΑΦΗΣ (Εμφανίζεται στη σελίδα των ευχαριστιών,
%% αν υπάρχει).
%
	\ackPlaceDate{Αθήνα, Φεβρουάριος 2021}
%
%% -------------------------------------------------------------------
%% ΗΜΕΡΟΜΗΝΙΑ ΕΞΕΤΑΣΗΣ
%
	\examinationDate{22α Φεβρουαρίου 2022}
%% -------------------------------------------------------------------
%% ΗΜΕΡΟΜΗΝΙΑ ΔΗΛΩΣΗΣ ΠΕΡΙ ΜΗ ΛΟΓΟΚΛΟΠΗΣ
%
	\declarationDate{10 Ιανουαρίου 2021}
%
%% -------------------------------------------------------------------
%% ΕΤΟΣ COPYRIGHT
%
	\copyrightYear{2021}
%
%% -------------------------------------------------------------------
%% ΟΝΟΜΑΤΕΠΩΝΥΜΟ 1ου ΕΞΕΤΑΣΤΗ
%
	\firstExaminer{\en{1st Examiner}}
%
%% -------------------------------------------------------------------
%% ΤΙΤΛΟΣ 1ου ΕΞΕΤΑΣΤΗ
%
	\firstExaminerTitle{\en{1st Examiner's Title}}
%
%% -------------------------------------------------------------------
%% ΟΝΟΜΑΤΕΠΩΝΥΜΟ 2ου ΕΞΕΤΑΣΤΗ
%
	\secondExaminer{\en{2nd Examiner}}
%
%% -------------------------------------------------------------------
%% ΤΙΤΛΟΣ 2ου ΕΞΕΤΑΣΤΗ
%
	\secondExaminerTitle{\en{2nd Examiner's Title}}
%%
%%
%%%%%%%%%%%%%%%%%%%%%%%%%%%%%%%%%%%%%%%%%%%%%%%%%%%%%%%%%%%%%%%%%%%%%%
%% THESIS COLORS: 
%%%%%%%%%%%%%%%%%%%%%%%%%%%%%%%%%%%%%%%%%%%%%%%%%%%%%%%%%%%%%%%%%%%%%%
%%
%% Χρώμα εξωφύλλου - κεφαλαίων
	\chaptercolor{gray!50!brown}
%%
%% Χρώμα παραρτημάτων
	\appendixcolor{brown!60!orange}
%%
%% Χρώμα υπερσυνδέσμων (αν έχει ενεργοποιηθεί η επιλογή "hyperref")
    \hyperlinkcolor{blue}
%%
%% Χρώμα τίτλου εργασίας στο εξώφυλλο (αν δεν έχει ενεργοποιηθεί 
%% η επιλογή "plain")
    \titlecolor{white}
%%
%% Χρώμα υποβάθρου (φόντου) τίτλου εργασίας στο εξώφυλλο (αν δεν έχει 
%% ενεργοποιηθεί η επιλογή "plain")
    \titlebackgroundcolor{gray!60!brown}  
%%
%%
%%%%%%%%%%%%%%%%%%%%%%%%%%%%%%%%%%%%%%%%%%%%%%%%%%%%%%%%%%%%%%%%%%%%%%
%% COVER PAGE IMAGE: 
%%%%%%%%%%%%%%%%%%%%%%%%%%%%%%%%%%%%%%%%%%%%%%%%%%%%%%%%%%%%%%%%%%%%%%
%%
%% Εικόνα εξωφύλλου (προαιρετική)
%% Στην περίπτωση κατά την οποία δεν είναι επιθυμητή η εισαγωγή εικόνας στο εξώφυλλο,
%% διαγράψτε την εντολή \coverpageimage, ή μετατρέψτε την σε σχόλιο (με "%")
%%
%% Σύνταξη:
%%          \coverpageimage{συντελεστής μεγέθυνσης}{όνομα αρχείου εικόνας [πλήρης διαδρομή]}
%%      ή
%%          \coverpageimage[tikz]{συντελεστής μεγέθυνσης}{εντολές TikZ}
%%          (στις εντολές μπορούν να περιλαμβάνονται και δηλώσεις \usetikzlibrary, κ.λπ.)
%%      
%% Παραδείγματα:
%%      - Χρήση εικόνας από το αρχείο "figures/rdf.png" με συντελεστή μεγέθυνσης 0.8:
%%          \coverpageimage{0.8}{figures/rdf.png}
%%      - Χρήση εικόνας TikZ με συντελεστή μεγέθυνσης 0.5:
%%          \coverpageimage[tikz]{0.5}{
%%              \draw[thick, gray] \foreach \x in {18,90,...,306} {
%%                  (\x:4) node{} -- (\x+72:4)
%%                  (\x:4) -- (\x:3) node{}
%%                  (\x:3) -- (\x+15:2) node{}
%%                  (\x:3) -- (\x-15:2) node{}
%%                  (\x+15:2) -- (\x+144-15:2)
%%                  (\x-15:2) -- (\x+144+15:2)
%%              };
%%          }
%%
%%     \coverpageimage{0.8}{figures/rdf.png}
%%
%%%%%%%%%%%%%%%%%%%%%%%%%%%%%%%%%%%%%%%%%%%%%%%%%%%%%%%%%%%%%%%%%%%%%%
%
% add custom hyphenation rules here
\hyphenation{ο-ποί-α} 
%re
%%%%
%
%
%%%%
\usepackage[utf8]{inputenc}
\usepackage[greek,english]{babel}
\usepackage{alphabeta}
\begin{document}

\maketitle

\beginfrontmatter
	
% Περίληψη
	\begin{abstract}
\en{ABSTRACT} ΔΙΠΛΩΜΑΤΙΚΗΣ

\begin{comment}
Ένα σύστημα ομότιμων κόμβων αποτελείται από ένα σύνολο αυτόνομων
υπολογιστικών κόμβων στο Διαδίκτυο, οι οποίοι συνεργάζονται με
σκοπό την ανταλλαγή δεδομένων. Στα συστήματα ομότιμων κόμβων που
χρησιμοποιούνται ευρέως σήμερα, η αναζήτηση πληροφορίας γίνεται με
χρήση λέξεων κλειδιών. Η ανάγκη για πιο εκφραστικές λειτουργίες,
σε συνδυασμό με την ανάπτυξη του Σημασιολογικού Ιστού, οδήγησε στα
συστήματα ομότιμων κόμβων βασισμένα σε σχήματα. Στα συστήματα αυτά
κάθε κόμβος χρησιμοποιεί ένα σχήμα με βάση το οποίο οργανώνει τα
τοπικά διαθέσιμα δεδομένα. Για να είναι δυνατή η αναζήτηση
δεδομένων στα συστήματα αυτά υπάρχουν δύο τρόποι. Ο πρώτος είναι
όλοι οι κόμβοι να χρησιμοποιούν το ίδιο σχήμα κάτι το οποίο δεν
είναι ευέλικτο. Ο δεύτερος τρόπος δίνει την αυτονομία σε κάθε
κόμβο να επιλέγει όποιο σχήμα θέλει και απαιτεί την ύπαρξη κανόνων
αντιστοίχισης μεταξύ των σχημάτων για να μπορούν να αποτιμώνται οι
ερωτήσεις. Αυτός ο τρόπος προσφέρει ευελιξία όμως δεν υποστηρίζει
την αυτόματη δημιουργία και τη δυναμική ανανέωση των κανόνων, που
είναι απαραίτητες για ένα σύστημα ομότιμων κόμβων.

Στόχος της διπλωματικής εργασίας είναι η ανάπτυξη ενός συστήματος
ομότιμων κόμβων βασισμένο σε σχήματα το οποίο (α) θα επιτρέπει μια
σχετική ευελιξία στην χρήση των σχημάτων και (β) θα δίνει την
δυνατότητα μετασχηματισμού ερωτήσεων χωρίς την ανάγκη διατύπωσης
κανόνων αντιστοίχισης μεταξύ σχημάτων, xρησιμοποιώντας κόμβους με
σχήματα \tl{RDF} που αποτελούν υποσύνολα-όψεις ενός βασικού
σχήματος (καθολικό σχήμα).

   \begin{keywords}
   Σύστημα ομότιμων κόμβων, Σύστημα ομότιμων κόμβων βασισμένο σε
   σχήματα, Σημασιολογικός Ιστός, \tl{RDF/S}, \tl{RQL}, \tl{Jxta}
   \end{keywords}

\end{comment}

\end{abstract}



\begin{abstracteng}
\en{ABSTRACT} ΔΙΠΛΩΜΑΤΙΚΗΣ στα Αγγλικά

\begin{comment}
\tl{A peer-to-peer system is a set of autonomous computing nodes
(the peers) which cooperate in order to exchange data. The peers
in the peer-to-peer systems that are widely used today, rely on
simple keyword selection in order to search for data. The need for
richer facilities in exchanging data, as well as, the evolution of
the Semantic Web, led to the evolution of the schema-based
peer-to-peer systems. In those systems every node uses a schema to
organize the local data. So there are two ways in order for data
search to be feasible. The first but not so flexible way implies
that every node uses the same schema. The second way gives every
node the flexibility to choose a schema according with its needs,
but on the same time requires the existence of mapping rules in
order for queries to be replied. This way though, doesn't offer
automatic creation and dynamic renewal of the mapping rules which
would be essential for peer-to-peer systems.}

\tl{This diploma thesis aims to the development of a schema-based
peer-to-peer system that allows a certain flexibility for schema
selection and on the same time enables query transformation
without the use of mapping rules. The peers use RDF schemas that
are subsets (views) of a big common schema called global schema.}

   \begin{keywordseng}
    \tl{Peer-to-peer, Schema-based peer-to-peer, Semantic Web, RDF/S, RQL, Jxta}
   \end{keywordseng}
\end{comment}

\end{abstracteng}
% Αφιέρωση
	\thesisDedication{στους γονείς μου}
% Ευχαριστίες
	%%%%%%%%%%%%%%%%%%%%%%%%%%%%%%%%%%%%%%%%%%%%%%%%%%%%%%%%%%%%%%%%%
%%
%% use the starred version of the "acknowledgements" environment
%% to omit signatures from this section, e.g.:
%% \begin{acknowledgements*} ... \end{acknowledgements*}
%% 
%%%%%%%%%%%%%%%%%%%%%%%%%%%%%%%%%%%%%%%%%%%%%%%%%%%%%%%%%%%%%%%%%
\begin{acknowledgements}
Θα ήθελα καταρχήν να ευχαριστήσω τον καθηγητή κ. Δημήτριο Σούντρη
για την επίβλεψη αυτής της διπλωματικής εργασίας και για την
ευκαιρία που μου έδωσε να την εκπονήσω στο Εργαστήριο Μικροεπεξεργαστών και Εργαστήριο Ψηφιακών Συστημάτων. Επίσης ευχαριστώ ιδιαίτερα τον υποψήφιο Δρ.
Μανώλη Κατσαραγάκη και τον υποψήδιο Δρ. Δημοσθένη Μασούρο για την καθοδήγησή του και την εξαιρετική
συνεργασία που είχαμε καθ'όλη τη διάρκεια της εκπόνησης της εργασίας. Τέλος θα ήθελα να ευχαριστήσω τους γονείς
μου για την καθοδήγηση και την ηθική συμπαράσταση που μου
προσέφεραν όλα αυτά τα χρόνια.
\end{acknowledgements}
% Πίνακας Περιεχομένων
	\tableofcontents
% Κατάλογος Σχημάτων
%    \listoffigures
% Κατάλογος Εικόνων
%	\listofillustrations
% Κατάλογος Πινάκων
%	\listoftables
% Πρόλογος
%	\begin{preface}
Στον πρόλογο αναφέρονται θέματα που δεν είναι επιστημονικά ή τεχνικά, όπως το πλαίσιο που διενεργήθηκε η εργασία, ο τόπος διεξαγωγής, το Εργαστήριο στο οποίο εκπονήθηκε κ.λπ. 
\end{preface}
	
\beginmainmatter

%%%%%%%%%%%%%%%%%%%%%%%%%%%%%%%%%%%%%%%%%%%%%%%%%%%%%
%% INCLUDE YOUR CHAPTERS/SECTIONS HERE
%%
% Εισαγωγή
	\chapter{Εισαγωγή} 
\InitialCharacter{Τ}α τελευταία χρόνια παρατηρείται εκτεταμένη διείσδυση της Μηχανικής Μάθησης σε κάθε κλάδο. Η Μηχανική μάθηση αποτελεί παράγοντα καινοτομίας σε ένα μεγάλο φάσμα εφαρμογών που περιλαμβάνει απο εμπορικά προιόντα έως και ιατρικές εφαρμογές. Η ευρύτητα αυτού του φάσματος ,σε συνδυασμό με το γεγονός ότι η απόδοση των μοντέλων που αναπτύσσονται με τεχνικές μηχανικής μάθησης είναι αρκετά καλή μοντελοποιώντας εφαρμογές που παραδοσιακά θα ήταν αρκετά περίπλοκο να μοντελοποιηθούν, ωθούν στην ραγδαία ανάπτυξη της. Ταυτόχρονα, στον κλάδο της αρχιτεκτονικής υπολογιστών η προόδος που προβλέπεται απο το νόμο του Moore φαίνεται ότι σταδιακά παύει να ακολουθεί το εκθετικό μοτίβο αύξησης που ακολουθούσε έως τώρα , ενώ το χάσμα επιδόσεων μεταξύ μνήμης και επεξεργαστή δεν εχει γεφυρωθεί ακόμα. Οι δύο αυτές τάσεις ,δηλαδή η εξέλιξη της Μηχανικής Μάθησης και τα ίδια τα προβλήματα που υπάρχουν στην Αρχιτεκτονικη Υπολογιστων, ωθούν προς μια συνδυαστική αξιοποιήση μεθόδων και τεχνικών , έτσι ώστε η μηχανική μάθηση να υποστηρίζεται αλλα κυρίως να υποστηρίζει την αρχιτεκτονική.


\begin{comment}
\InitialCharacter{Ο} Παγκόσμιος Ιστός test αποτελεί χώρο διακίνησης τεράστιου όγκου
πληροφοριών. Ωστόσο, η συντριπτική πλειοψηφία των πληροφοριών του
Ιστού, είναι προσανατολισμένη προς τον άνθρωπο-χρήστη και δεν
είναι κατανοητή από τις εφαρμογές. Για να αξιοποιηθεί λοιπόν η
διαθέσιμη πληροφορία και να γίνει πιο εύκολη η ανταλλαγή και η
επεξεργασία της, ο Παγκόσμιος Ιστός εξελίσσεται στο Σημασιολογικό
Ιστό.

Ο Σημασιολογικός Ιστός, είναι μια εξέλιξη του σημερινού Ιστού,
μέσα στον οποίο δίνεται καλά ορισμένο νόημα στην πληροφορία που
διακινείται, διευκολύνοντας τη συνεργασία μεταξύ υπολογιστή και
ανθρώπου \cite{LiArTs13}. Πιο συγκεκριμένα δίνει τη
δυνατότητα καλύτερης πρόσβασης σε μεγάλο όγκο πηγών πληροφορίας,
καθώς και πιο αποτελεσματικής διακίνησης των πληροφοριών,
χρησιμοποιώντας δεδομένα που τις περιγράφουν και ονομάζονται
$``$μεταδεδομένα$"$. Η καλύτερη γνώση της σημασίας, της χρήσης και
της ποιότητας των πηγών διευκολύνει σημαντικά τη δυνατότητα
πρόσβασης σε πηγές του Ιστού και την αυτόματη επεξεργασία του
περιεχομένου που υπάρχει διαθέσιμο στο Διαδίκτυο βάσει του
νοήματος και όχι μόνο της μορφής της πληροφορίας.

Ένα από τα πιο βασικά θέματα για την ανάπτυξη του Σημασιολογικού
Ιστού είναι το να μπορούν οι υπολογιστές να ανταλλάσσουν δεδομένα
 μεταξύ εφαρμογών. Σε ένα ανοιχτό περιβάλλον όπως
είναι ο Σημασιολογικός Ιστός χρειάζεται ένα ευέλικτο και δυναμικό
μοντέλο ανταλλαγής δεδομένων όπως είναι τα συστήματα ομότιμων
κόμβων \en{(Peer-to-Peer systems)}.

Ένα σύστημα ομότιμων κόμβων αποτελείται από ένα σύνολο αυτόνομων
υπολογιστικών κόμβων, οι οποίοι συνεργάζονται με σκοπό την
ανταλλαγή δεδομένων. Τα συστήματα ομότιμων κόμβων που
χρησιμοποιούνται ευρέως σήμερα κυρίως για την ανταλλαγή αρχείων
μουσικής, έχουν πολύ μικρές δυνατότητες διαχείρισης δεδομένων. Η
αναζήτηση πληροφορίας στα περισσότερα από αυτά γίνεται με χρήση
λέξεων κλειδιών \en{(keyword-based search)}.

Η ανάγκη για πιο εκφραστικές λειτουργίες, σε συνδυασμό με την
ανάπτυξη του Σημασιολογικού Ιστού, οδήγησε στα συστήματα ομότιμων
κόμβων που είναι βασισμένα σε σχήματα \en{(schema-based
peer-to-peer systems)}. Στα συστήματα αυτά κάθε κόμβος
χρησιμοποιεί ένα σχήμα με βάση το οποίο οργανώνει τα τοπικά
διαθέσιμα δεδομένα. Οι τεχνολογίες του Σημασιολογικού Ιστού δίνουν
τη δυνατότητα οργάνωσης των δεδομένων μέσω σχημάτων που τα
περιγράφουν.

Το πλαίσιο \en{RDF} είναι ένα τέτοιο εργαλείο αναπαράστασης
μεταδεδομένων. Σε ένα \en{RDF} αρχείο ορίζονται δηλώσεις για
αντικείμενα του Ιστού όπως σελίδες, συγγραφείς, προγράμματα κ.τ.λ.
Μια επέκταση του πλαισίου \en{RDF} είναι το \en{RDF Schema} το
οποίο παρέχει μηχανισμούς περιγραφής σχετικών αντικειμένων του
Ιστού καθώς και των σχέσεων μεταξύ τους. Το \en{RDF Schema}
βασίζεται σε κλάσεις και ιδιότητες έννοιες γνωστές από το χώρο των
Αντικειμενοστρεφών συστημάτων. Η βασική διαφορά είναι ότι στο
πλαίσιο \en{RDF} οι ιδιότητες ορίζονται ανεξάρτητα από τις
κλάσεις.

Χρησιμοποιώντας λοιπόν τις τεχνολογίες του Σημασιολογικού Ιστού
μπορούμε να δημιουργήσουμε συστήματα ομότιμων κόμβων με αυξημένη
διαλειτουργικότητα τα οποία θα ανταλλάσσουν μεταξύ τους πληροφορία
με νόημα και θα έχουν τη δυνατότητα διατύπωσης ερωτήσεων πιο
εκφραστικών από αυτές που βασίζονται σε λέξεις κλειδιά.
\end{comment}

\section{Αντικείμενο της διπλωματικής}
Η διπλωματική εργασία αυτή ασχολείται με τη χρήση τεχνητών νευρωνικών δικτύων για την χρονοδρομολόγηση σελίδων μνήμης σε σύγχρονα συστήματα που διαθέτουν δύο είδη κύριας μνήμης , τόσο συμβατική \en{DRAM} όσο και ένα σχετικά νέο είδος μνήμης την \en{Persistent Memory}

Είναι γνωστό ότι τα σύγχρονα υπολογιστικά συστήματα σχεδιάζονται χρησιμοποιώντας ετερογενή συστατικά μνήμης. Αυτές οι μνήμες συχνά  είτε εξυπηρετούν στην αύξηση της χωρητικότητας της κύριας μνήμης, δηλαδή σαν επέκταση της \en{DRAM}, είτε αξιοποιούνται ως κρυφές-μνήμες(\en{caches}) της κύριας μνήμης. Αυτά τα υβριδικά συστήματα μνήμης απαιτούν εκ φύσεως κάποιους σχεδιαστικούς συμβιβασμούς. Συνήθως η μακρύτερη στην ιεραρχία μνήμη, δηλαδή μνήμη που βρίσκεται πιο μακρία απο την Επεξεργαστική Μονάδα, έχει μεγαλύτερη χωρητικότητα αποθήκευσης αλλά ταυτόχρονα έχει και μεγαλύτερη καθυστέρηση (\en{latency}) και μειωμένο εύρος ζώνης (\en{bandwidth}).

Στη συγκεκριμένη διπλωματική εργασία θα περιοριστούμε σε υπολογιστικά συστήματα τα οποία αξιοποιούν την \en{Persistent Memory} ως επέκταση της κύριας μνήμης. Για το σχεδιασμό αυτών των συστημάτων είναι ιδιαίτερα σημαντικό να ληφθούν υπόψιν το μεγαλύτερο \en{latency} και το μειωμένο \en{bandwidth} που παρουσίαζει αυτή η μνήμη σε σχέση με τη \en{DRAM}. Στη συγκεκριμένη περίπτωση αυτό που μας ενδιαφέρει η αποδοτική υλοποιήση ενός Χρονοδρομολογητή Σελίδων Μνήμης (\en{Page Scheduler}) , δηλαδή η αποδοτική υλοποίηση της μονάδας που αναλαμβάνει τη διαχείριση της μνήμης του λειτουργικού συστήματος αλλά και τον εν εκτελέση προγραμμάτων. Ο χρονοδρομολογητής σελίδων θα είναι υπεύθυνος για τη μεταφορά σελίδων μνήμης απο και προς τα διάφορα ετερογενή συστατικά μνήμης που απαρτίζουν το σύστημα μας . Κύριο σκοπό θα έχει οι σελίδες μνήμης που θεωρούνται \textit{\en{Hot}}, δηλαδή προσπελαύνονται συχνά, να βρίσκονται στα υψηλής απόδοσης στοιχεία μνήμης που διαθέτει το σύστημα μας, δηλαδή στη \en{DRAM}, ενώ λιγότερο σημαντικές σελίδες μνήμης \textit{\en{Cold}} να βρίσκονται στην \en{Persistent Memory}. 

\section{Κίνητρο και συναφείς προσεγγίσεις}
Πολλοί ερευνητές έως τώρα έχουν αποπειραθεί να δώσουν μια λύση στο πρόβλημα της κατηγοριόποισης των σελίδων μνήμης και στην κατάλληλη τοποθέτηση τους στα διάφορα ετερογενή στοιχεία μνήμης. Πρόκειται σίγουρα για μία δύσκολη διαδικασία καθώς πρέπει να ληφθούν υπόψιν και το μοτίβο προσπέλασης μνήμης που ακολουθεί μία εφαρμογή όσο και οι παράμετροι εκτέλεσης της εφαρμογής (μέγεθος του \en{input}, \en{strong/weak scaling etc.}). Οι περισσότεροι ερευνητές έχουν προτείνει λύσεις στο παραπάνω πρόβλημα που μπορούν να ενσωματωθούν στο επίπεδο του \en{hardware} , των \en{Compilers} ,του Λειτουργικού Συστήματος ,του περιβάλλοντος εκτέλεσης. \cite{Chou2017BATMANTF,10.1145/2901318.2901344,10.1145/2926697.2926702,10.1145/3132402.3132418,10.1145/3241624.2926702,10.5555/3291656.3291698,DBLP:journals/corr/WuHL17}. Συχνά οι ερευνητές σε αυτές τις προσεγγίσεις χρησιμοποιούν πληροφορίες που σχετίζονται μονάχα με το ιστορικό προσπέλασης των σελίδων μνήμης. Συγκεκριμένα οι σύγχρονες τεχνικές που χρησιμοποιούνται στη δυναμική διαχείριση σελίδων σε επίπεδο συστήματος χρησιμοποιούν την πρόσφατη παρατηρούμενη συμπεριφορά των σελίδων προκειμένου να παρθούν αποφάσεις την μελλοντική τοποθέτηση τους.

\section{Προσέγγιση και Συνεισφορές}
Σε αυτή τη διπλωματική εργασία θα ακολουθηθεί η συλλογιστική πορεία των άρθρων \emph{\en{Learning Memory Access Patterns}} \cite{hashemi2018learning} και \emph{\en{Kleio: A hybrid Memory Page Scheduler}} \cite{10.1145/3307681.3325398}
\\

Σκοπός μας είναι η μελέτη και κατασκευή ενός Χρονοδομολογητή Σελίδων Μνήμης (\en{Hybrid Memory Page Scheduler}) χρησιμοποιώντας Μηχανική Μάθηση, ο  οποίος θα επιτυγχάνει καλύτερη απόδοση απο τις σύγχρονες μεθόδους οι οποίες στηρίζονται αποκλειστικά στην ιστορική παρατήρηση προσβάσεων στη μνήμη των εφαρμογών. Θα προσπαθήσουμε να απαντήσουμε σε ερωτήσεις που αφορούν το πως θα πετύχουμε μια αποδοτική λύση , δηλαδή πως θα καταφέρουμε να τοποθετήσουμε όσο πιο ιδανικά γίνεται τα σωστά δεδομένα σε επίπεδο σελίδων (\emph{\textbf{4KB}}) στις σωστές θέσεις. Ενώ ταυτόχρονα αναζητούμε και μια εφικτή λύση, χρησιμοποιώντας περιορισμένους επεξεργαστικούς πόρους για την κατασκευή των μοντέλων μηχανικής μάθησης που θα χρειαστούμε. Ακουλουθεί η αναλυτική περιγραφή των ζητημάτων που θα μελετηθούν σε αυτήν διπλωματική
\\
\\
\begin{itemize}
    \item Το χάσμα απόδοσης που υπάρχει στις σύγχρονες προσπάθειες αντιμετώπισης του προβλήματος. Λόγω της πρόσφατης άφιξης των νέων αυτών μνημών , ο τρόπος που αντιμετωπίζεται το πρόβλημα είναι αρκετά απλοϊκός με αποτέλεσμα να μην είναι αρκετά αποδοτικός όταν οι εφαρμογές προς εκτέλεση παρουσιάζουν΄περίπλοκα μοτίβα προσβάσεων μνήμης.
    \item Η κατάλληλη επιλογή της συχνότητας του Χρονοδρομολογητη. Θα διαπιστώσουμε αναλύοντας τις προσβάσεις μνήμης διάφορων μετροπρογραμμάτων κατά πόσο επηρεάζει την απόδοση η συχνότητα με την οποία ο Χρονοδρομολογητης αποφασίζει να μεταφέρει σελίδες μεταξύ των διάφορων στοιχείων μνήμης του συστήματος
    \item Χρονοδρομολογηση με χρήση Μηχανικής Μάθησης. Θα διαπιστώσουμε ότι τα επαναληπτικά Νευρωνικά Δίκτυα (\en{RNN}) μπορούν να αξιοποιηθούν κατάλληλα για το πρόβλημα του \en{Scheduling} όπως άλλωστε παρατηρήθηκε και στα άρθρα  \cite{hashemi2018learning,10.1145/3307681.3325398}. Θα προσπαθήσουμε να προσαρμόσουμε το πρόβλημα όσο πιο κατάλληλα γίνεται ώστε να επιτύχουμε ακριβείς προβλέψεις των Νευρωνικών Δικτύων χωρίς όμως να αγνοούμε την υπολογιστική και χωρική πολυπλοκότητα της προσέγγισης μας.
    \item Κατασκευή \en{Page-Scheduler} . Θα προσπαθήσουμε αναλογιζόμενοι διάφορες κρίσιμες μετρικές να κατασκευάσουμε έναν \en{Page-Scheduler} ο οποίος θα χρησιμοποιεί τόσο τις σύγχρονες μεθόδους \en{History-Based} Χρονοδρομολόγησης,όσο και Τεχνητή Νοημόσυνη βασισμένη στα Επαναληπτικά Νευρωνικά Δίκτυα προκειμένου να επιτευχθεί μια αύξηση της απόδοσης.
\end{itemize}




\section{Οργάνωση του τόμου}

\begin{comment}
Η εργασία αυτή είναι οργανωμένη σε επτά κεφάλαια: Στο Κεφάλαιο 2
δίνεται το θεωρητικό υπόβαθρο των βασικών τεχνολογιών που
σχετίζονται με τη διπλωματική αυτή. Αρχικά περιγράφονται τα δίκτυα
ομότιμων κόμβων, στη συνέχεια το πλαίσιο \en{RDF} και τέλος
δίνεται μια μελέτη των γλωσσών ερωτήσεων για \en{RDF}. Στο
Κεφάλαιο 3 αρχικά περιγράφονται οι σχετικές με το θέμα εργασίες
και στη συνέχεια δίνεται ο στόχος της συγκεκριμένης εργασίας. Στο
Κεφάλαιο 4 παρουσιάζεται η ανάλυση και η σχεδίαση του συστήματος,
δηλαδή η περιγραφή των υποσυστημάτων και των εφαρμογών του. Η
περιγραφή της υλοποίησης του συστήματος, με ανάλυση των βασικών
αλγορίθμων καθώς και λεπτομέρειες σχετικά με τις πλατφόρμες και τα
προγραμματιστικά εργαλεία που χρησιμοποιήθηκαν δίνεται στο
Κεφάλαιο 5. Στο Κεφάλαιο 6 παρουσιάζεται ο έλεγχος καλής
λειτουργίας του συστήματος με βάση ένα συγκεκριμένο σενάριο
χρήσης. Τέλος στο Κεφάλαιο 7 δίνεται η συνεισφορά αυτής της
διπλωματικής εργασίας, καθώς και μελλοντικές επεκτάσεις.
\end{comment}
% Μέρη/Κεφάλαια
%	\part{Θεωρητικό Μέρος}
	\chapter{\en{Persistent Memory}}

\section{Persistent memory θεωρία}

\section{\en{Page Scheduling σε Hybrid Memory Systems}}
%	\include{body_matter/chap3}
%    \part{Πρακτικό Μέρος} 
%	\include{body_matter/chap4}
%	\include{body_matter/chap5}
%	\include{body_matter/chap6}
%	\include{body_matter/chap7}
%	\include{body_matter/chap8}
%    \part{Επίλογος}
%	\include{body_matter/chap9}
% Παραρτήματα
%	\appendices
%	\include{back_matter/appA}
%	\include{back_matter/appB}
%    \include{back_matter/appC}
%   \include{back_matter/appD}
%    \include{back_matter/appE} 
% Βιβλιογραφία - Αναφορές
	\bibliography{references}
% Συντομογραφίες - Αρκτικόλεξα - Ακρωνύμια
%	\includeabbreviations{back_matter/abbreviations}
% Γλωσσάριο
%	\includeglossary{back_matter/glossary}
%%%%%%%%%%%%%%%%%%%%%%%%%%%%%%%%%%%%%%%%%%%%%%%%%%%%
% Ευρετήριο Όρων
	\printindices
%
%%%%%%%%%%%%%%%%%%
%%%%%%%%%%%%%%%%%%

%% Δημιουργία ετικετών CD:

	\definecdlabeloffsets{0}{-0.65}{0}{0.55} % upper label x offset [cm] (default=0) /  upper label y offset [cm] (default=0) /  lower label x offset [cm] (default=0) /  lower  label y offset [cm] (default=0) -- For Q-Connect KF01579 labels use the following offset values: {0}{-0.65}{0}{0.55}


    \createcdlabel{Hybrid Mem sceduler \\ Based on Machine Learning \en{RDF}}{Κωνσταντίνος Σταυρακάκης}{Φεβρουάριος}{2021}{8}
	\createcdlabel{Hybrid Mem sceduler \\ Based on Machine Learning \en{RDF}}{Κωνσταντίνος Σταυρακάκης}{Φεβρουάριος}{2021}{8} % τίτλος διπλωματικής / όνομα συγγραφέα / μήνας / έτος / εύρος περιοχής τίτλου σε cm (προτεινόμενη τιμή: 8) 

%%σ
%% Δημιουργία εξωφύλλου θήκης CD:

	\createcdcover{Hybrid Mem sceduler \\ Based on Machine Learning \en{RDF}}{Κωνσταντίνος Σταυρακάκης}{ΦΕΒΡΟΥΑΡΙΟΣ}{10} % τίτλος πτυχιακής / όνομα συγγραφέα / μήνας / έτος / εύρος περιοχής τίτλου σε cm (προτεινόμενη τιμή: 10) 

%%
%
\end{document}

%%%%%%%%%%%%%%%%%%%%%%%%%%%%%%%%%%%%%%%%%%%%%%%%%%%%
